\section*{Project Summary}


\paragraph{Intellectual Merit.} 
If successful, the proposed activities will allow the estimation of
genomic bio-diversity for a fraction of the current costs of labor and
genome sequencing. The proposal uses a number of innovative and novel
algorithmic and statistical techniques, and describes the first
systematic study of the feasibility of computing the genomic distance using only a small, random fraction of the genome.  If successful, the project will advance the field by providing a simple and inexpensive protocol for measuring biodiversity with higher sensitivity than is currently achievable. The investigators
have a strong history of prior research in related fields, but have
complementary expertise, in evolution and phylogenetic reconstruction
(Mirarab), and computational population genomics (Bafna).

\paragraph{Broader Impacts.}
Much of the planet's biodiversity is in the least developed and
poorest places on the planet. Rapid environmental change and
anthropogenic activities are degrading this bio-diversity and have
potentially severe and lasting impact on all people, including in the
U.S. The proposed tools enable the cataloging and measurement of
bio-diversity through inexpensive sequencing, and easy-to-adopt
laboratory protocols. While the proposal focuses on computational
tools, the experiments will demonstrate proof of concept and the
viability of larger sampling studies with a view towards deploying
them where they are most needed. The proposal will also help train
scientists who are better aware of the impact of rapid environmental
changes on ecology, and the role of genomics and computation in
investigating and alleviating these impacts. Through outreach activities, trainees will include undergraduate students from under-represented communities.

\paragraph{Keywords:} bioinformatics; computational genomics; computational evolution.
